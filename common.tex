\usepackage[top=1in,left=1in,right=1in,bindingoffset=5mm,twoside]{geometry}

\usepackage{amssymb}
\usepackage{enumitem}
\usepackage{framed}
\usepackage[framed]{ntheorem}
\usepackage{dsfont}
\usepackage{bm}
\usepackage{quoting}
\usepackage{setspace}
\usepackage{xifthen}
\usepackage{textcomp}
\usepackage{nameref}
\usepackage{tikz}
\usepackage{graphicx}

\setlength{\parskip}{.5ex}

\theoremstyle{margin}
\theorembodyfont{\normalfont}
\theoremseparator{.}
\theorempreskipamount=\baselineskip
\theorempostskipamount=-1.5em

\theoremprework{\begin{onehalfspace}} \theorempostwork{\end{onehalfspace}}
\newtheorem{defn}{Definition}  %[chapter]

\theoremprework{\begin{onehalfspace}} \theorempostwork{\end{onehalfspace}}
\newtheorem{example}[defn]{Example}

\theoremprework{\begin{onehalfspace}} \theorempostwork{\end{onehalfspace}}
\newtheorem{axiom}[defn]{Axiom}

\theoremprework{\begin{onehalfspace}} \theorempostwork{\end{onehalfspace}}
\newtheorem{corol}[defn]{Corollary}

\theoremprework{\begin{onehalfspace}} \theorempostwork{\end{onehalfspace}}
\newtheorem{thmex}[defn]{Example Theorem}

\theoremheaderfont{\scshape}
\theoremindent\parindent
\theoremprework{\begin{onehalfspace}} \theorempostwork{\end{onehalfspace}}
\newtheorem*{proofex}[defn]{Example Proof}
\theoremheaderfont{\normalfont\bfseries}
\theoremindent0cm

\theoremstyle{plain}
\theorembodyfont{\normalfont}
%\theoremindent=\parindent
%\theoremseparator{.}

\theorempreskipamount=-2em
\theorempostskipamount=-.75em

\theoremprework{\begin{onehalfspace}} \theorempostwork{\end{onehalfspace}}
\newframedtheorem{thm}[defn]{Theorem}
%\newtheorem{thm}[defn]{Theorem}

\theoremprework{\begin{onehalfspace}} \theorempostwork{\end{onehalfspace}}
\newframedtheorem{lemma}[defn]{Lemma}

\theoremprework{\begin{onehalfspace}} \theorempostwork{\end{onehalfspace}}
\newframedtheorem{stmt}[defn]{Statement}

\theoremprework{\begin{onehalfspace}} \theorempostwork{\end{onehalfspace}}
\newframedtheorem{exer}[defn]{Exercise}

\theoremprework{\begin{onehalfspace}} \theorempostwork{\end{onehalfspace}}
\newframedtheorem{progexer}[defn]{Programming Exercise}


\newenvironment{exersoln}[1]{\textbf{Exercise #1 Solution.}\begin{quote}}{\end{quote}}
\newenvironment{stmtsoln}[1]{\textbf{Statement #1 Proof.}\begin{quote}}{\end{quote}}
\newenvironment{thmsoln}[1]{\textbf{Theorem #1 Proof.}\begin{quote}}{\end{quote}}
\newcommand\answer{\par\textbf{\emph{Answer}.\ }}
\newcommand\proof{\par\textbf{\emph{Proof}.\ }}

\newenvironment{exenumerate}{\begin{enumerate}[label=(\alph*),leftmargin=3em]}{\end{enumerate}}
%\newenvironment{discussion}{\begin{quoting}[vskip=0pt]}{\end{quoting}}
\newenvironment{discussion}{\begin{leftbar}\noindent}{\end{leftbar}}
%\newenvironment{discussion}{}{}

\newcommand\defterm[1]{\emph{\textbf{#1}}}  % a term being defined
\newcommand\hint{\\ \emph{Hint:}\ }
\newcommand*{\happymac}{\hfill\includegraphics[height=2em]{happymac.png}\\}

\newcommand\emptystring\epsilon
\newcommand\definedas{\ \equiv\ }
\newcommand\implies\to
\newcommand*{\qed}{\hfill\ensuremath{\square}}

\newcommand\naturals{\mathds{N}}

\newcommand\setbuildsepname{colon}
\newcommand\setbuildsep{:}
\newcommand\setdef[1]{\left\lbrace #1 \right\rbrace}
\newcommand\setbuild[2]{\left\lbrace #1\ \setbuildsep\ #2 \right\rbrace}
\newcommand\union\cup
\newcommand\intersection\cap
\newcommand\setunion[2]{#1\,\union\,#2}
\newcommand\setintersection[2]{#1\,\intersection\,#2}
\newcommand\setconcat[2]{#1\,\circ\,#2}
\newcommand\setcomplement[1]{\overline{#1}}
\newcommand\setstar[1]{#1^{*}}
\newcommand\powerset[1]{\mathcal{P}(#1)}
\newcommand\setproduct[2]{#1 \times #2}

\newcommand\langof[1]{\mathcal{L}(#1)}

\newcommand\stringrev[1]{#1^{\mathcal{R}}}


