\documentclass[10pt]{article}

%\textwidth=7in
%\textheight=9.5in
%\topmargin=-1in
%\headheight=0in
%\headsep=.5in
%\hoffset=-1in

\usepackage[top=1in,left=1in,right=1in,bindingoffset=5mm,twoside]{geometry}

\usepackage{amssymb}
\usepackage{enumitem}
\usepackage{framed}
\usepackage[framed]{ntheorem}
\usepackage{dsfont}
\usepackage{bm}
\usepackage{quoting}
\usepackage{setspace}
\usepackage{xifthen}
\usepackage{textcomp}
\usepackage{nameref}
\usepackage{tikz}
\usepackage{graphicx}

\setlength{\parskip}{.5ex}

\theoremstyle{margin}
\theorembodyfont{\normalfont}
\theoremseparator{.}
\theorempreskipamount=\baselineskip
\theorempostskipamount=-1.5em

\theoremprework{\begin{onehalfspace}} \theorempostwork{\end{onehalfspace}}
\newtheorem{defn}{Definition}  %[chapter]

\theoremprework{\begin{onehalfspace}} \theorempostwork{\end{onehalfspace}}
\newtheorem{example}[defn]{Example}

\theoremprework{\begin{onehalfspace}} \theorempostwork{\end{onehalfspace}}
\newtheorem{axiom}[defn]{Axiom}

\theoremprework{\begin{onehalfspace}} \theorempostwork{\end{onehalfspace}}
\newtheorem{corol}[defn]{Corollary}

\theoremprework{\begin{onehalfspace}} \theorempostwork{\end{onehalfspace}}
\newtheorem{thmex}[defn]{Example Theorem}

\theoremheaderfont{\scshape}
\theoremindent\parindent
\theoremprework{\begin{onehalfspace}} \theorempostwork{\end{onehalfspace}}
\newtheorem*{proofex}[defn]{Example Proof}
\theoremheaderfont{\normalfont\bfseries}
\theoremindent0cm

\theoremstyle{plain}
\theorembodyfont{\normalfont}
%\theoremindent=\parindent
%\theoremseparator{.}

\theorempreskipamount=-2em
\theorempostskipamount=-.75em

\theoremprework{\begin{onehalfspace}} \theorempostwork{\end{onehalfspace}}
\newframedtheorem{thm}[defn]{Theorem}
%\newtheorem{thm}[defn]{Theorem}

\theoremprework{\begin{onehalfspace}} \theorempostwork{\end{onehalfspace}}
\newframedtheorem{lemma}[defn]{Lemma}

\theoremprework{\begin{onehalfspace}} \theorempostwork{\end{onehalfspace}}
\newframedtheorem{stmt}[defn]{Statement}

\theoremprework{\begin{onehalfspace}} \theorempostwork{\end{onehalfspace}}
\newframedtheorem{exer}[defn]{Exercise}

\theoremprework{\begin{onehalfspace}} \theorempostwork{\end{onehalfspace}}
\newframedtheorem{progexer}[defn]{Programming Exercise}


\newenvironment{exersoln}[1]{\textbf{Exercise #1 Solution.}\begin{quote}}{\end{quote}}
\newenvironment{stmtsoln}[1]{\textbf{Statement #1 Proof.}\begin{quote}}{\end{quote}}
\newenvironment{thmsoln}[1]{\textbf{Theorem #1 Proof.}\begin{quote}}{\end{quote}}
\newcommand\answer{\par\textbf{\emph{Answer}.\ }}
\newcommand\proof{\par\textbf{\emph{Proof}.\ }}

\newenvironment{exenumerate}{\begin{enumerate}[label=(\alph*),leftmargin=3em]}{\end{enumerate}}
%\newenvironment{discussion}{\begin{quoting}[vskip=0pt]}{\end{quoting}}
\newenvironment{discussion}{\begin{leftbar}\noindent}{\end{leftbar}}
%\newenvironment{discussion}{}{}

\newcommand\defterm[1]{\emph{\textbf{#1}}}  % a term being defined
\newcommand\hint{\\ \emph{Hint:}\ }
\newcommand*{\happymac}{\hfill\includegraphics[height=2em]{happymac.png}\\}

\newcommand\emptystring\epsilon
\newcommand\definedas{\ \equiv\ }
\newcommand\implies\to
\newcommand*{\qed}{\hfill\ensuremath{\square}}

\newcommand\naturals{\mathds{N}}

\newcommand\setbuildsepname{colon}
\newcommand\setbuildsep{:}
\newcommand\setdef[1]{\left\lbrace #1 \right\rbrace}
\newcommand\setbuild[2]{\left\lbrace #1\ \setbuildsep\ #2 \right\rbrace}
\newcommand\union\cup
\newcommand\intersection\cap
\newcommand\setunion[2]{#1\,\union\,#2}
\newcommand\setintersection[2]{#1\,\intersection\,#2}
\newcommand\setconcat[2]{#1\,\circ\,#2}
\newcommand\setcomplement[1]{\overline{#1}}
\newcommand\setstar[1]{#1^{*}}
\newcommand\powerset[1]{\mathcal{P}(#1)}
\newcommand\setproduct[2]{#1 \times #2}

\newcommand\langof[1]{\mathcal{L}(#1)}

\newcommand\stringrev[1]{#1^{\mathcal{R}}}




\title{Written Work - Mathematical Preliminaries: Sets} 	% TITLE OF THESE EXERCISES/PROOFS
\author{Nadeem Abdul Hamid}      			% YOUR NAME HERE
\date{\today}							% CHANGE IF YOU WANT

\begin{document}

\maketitle

\begin{exersoln}{16}   % 5 is the number of the exercise
	If $I = \setdef{1, 3, 9, 12}$ and $G = \setbuild{x \in I}{x~\textrm{is greater than 6}}$, what specifically 
	are the elements of $G$?
	
	\answer    % precede your answers with \answer, proofs with \proof
	Give your answer here.
	\qed		% for coolness
\end{exersoln}

\begin{exersoln}{19}
	Determine whether each of the following is true or false.
\begin{exenumerate}
\item $\emptyset \in \emptyset$
\item $\emptyset \subseteq \emptyset$ 
\item $\emptyset \in \setdef{\emptyset}$
\item $\emptyset \subseteq \setdef{\emptyset}$
\item $\setdef{a, b} \in \setdef{a, b, c, \setdef{a, b}}$
\item $\setdef{a, b} \subseteq \setdef{a, b, c, \setdef{a, b}}$
\end{exenumerate}

	\answer
	Give your answer here.
\end{exersoln}

\begin{exersoln}{21}
	For each of the operations in Definition 20, give a description in simple English of the resulting set, e.g. ``$\setunion{A}{B}$ contains \ldots"

	\answer
	Give your answer here.
\end{exersoln}

\begin{thmsoln}{26 (DeMorgan's Laws)}
	If $A$ and $B$ are sets,
\[\begin{array}{ll} 
\textrm{(a)} &  \setcomplement{\setunion{A}{B}} = \setintersection{\setcomplement{A}}{\setcomplement{B}}
\\
\textrm{(b)} & \setcomplement{\setintersection{A}{B}} = \setunion{\setcomplement{A}}{\setcomplement{B}}
\end{array}\]

	\answer
	Give your answer here.
\end{thmsoln}

\begin{stmtsoln}{27}
	If $A$, $B$, and $C$ are sets,
\[\begin{array}{ll} 
\textrm{(a)} &  A - (\setunion{B}{C}) = \setunion{(A-B)}{(A-C)}
\\
\textrm{(b)} & A - (\setintersection{B}{C}) = \setintersection{(A-B)}{(A-C)}
\end{array}\]

	\answer
	Give your answer here.
\end{stmtsoln}




%\begin{exersoln}{#}
%	
%	\answer
%	Give your answer here.
%\end{exersoln}



\end{document}