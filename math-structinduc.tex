\section{Structural Induction}

\begin{discussion}
This section may be skipped until you are ready to begin the section on~\nameref{sec:regexp} (page~\pageref{sec:regexp}).
\end{discussion}


We have previously mentioned \emph{inductive definitions} of a set. An inductive, or recursive, definition allows you to use the principle of structural induction to show some property is true of every item in the set. (Mathematical induction is simply a very specialized form of structural induction, based on the recursive definition of the set of natural numbers.)

\begin{defn}[Simple algebraic expressions]
We provide a simple, recursive, definition of algebraic expressions. To keep things simple, we restrict ourselves to only two binary operators, $+$ and $*$, and parentheses. 

The set of \defterm{algebraic expressions}, denoted $Expr$, is defined by:
\begin{enumerate}
\item $\underline{a} \in Expr$ for any single-letter $a$.
\item For any $x, y \in Expr$, \underline{$x + y$} and \underline{$x * y$} are also in $Expr$.
\item For any $x \in Expr$, \underline{$(x)$} is also in $Expr$.
\end{enumerate}
\end{defn}

\begin{discussion}
Based on the definition of $Expr$ above, we can prove properties using a corresponding principle of structure induction:
\\~
\end{discussion}

\begin{axiom}[Structural Induction on Algebraic Expressions]
Given a property or statement that depends on an algebraic expression, $P(x)$, in order to show that $P(x)$ is true for all $x \in Expr$, it is sufficient to show (3 cases, corresponding to the three cases of the definition):

\begin{enumerate}
\item $P(\underline{a})$ is true for any single-letter $a$.
\item For any $x, y \in Expr$, if $P(x)$ and $P(y)$ are true, then $P(\,\underline{x + y}\,)$ and $P(\,\underline{x * y}\,)$ are true.\footnote{The assumptions $P(x)$ and $P(y)$ in this case are the \emph{two} inductive hypotheses.}
\item For every $x \in Expr$, if $P(x)$ is true, then $P(\,\underline{(x)}\,)$ is true.
\end{enumerate}

\end{axiom}

\begin{thm}
For any algebraic expression $x \in Expr$, $x$ has equal numbers of left and right parentheses.
\end{thm}

\begin{thm}
For any algebraic expression $x \in Expr$, $x$ is made up of an odd number of symbols.
\end{thm}
